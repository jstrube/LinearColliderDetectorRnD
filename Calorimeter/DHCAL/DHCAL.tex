\section{Resistive Plate Chambers}
\subsection{List of participating institutes}
The following institutes are currently participating in the DHCAL efforts:
\begin{itemize}
\item Argonne National Laboratory -- member of CALICE
\item Boston University -- member of CALICE
\item COE College (Iowa)
\item University of Iowa -- member of CALICE
\item Shanghai Jiao Tong University -- SJTU (in discussion)
\item University of Science and Technology of China -- USTC (in discussion)
\end{itemize}

\subsection{Description of the DHCAL}
The Digital Hadron Calorimeter or DHCAL uses Resistive Plate Chambers (RPCs) as active elements. The chambers are read out with \unit[$1 \times 1$]{$^2$} pads and 1-bit (digital) resolution. A small-scale prototype was assembled and tested in the Fermilab test beam in 2007 to validate the concept.
Based on the success of the small-scale test [1-6], a large prototype with up to 54 layers and close to 500,000 readout channels was built in 2008 -- 2011. Each layer measured approximately \unit[$96 \times 96$]{$cm^2$} and was equipped with three chambers, stacked vertically on top of each other. 
For tests with particle beams the DHCAL layers were inserted into a main stack of 38 or 39 layers, followed by a tail catcher with up to 15 layers. For the tests performed at Fermilab the main stack contained steel absorber plates. At CERN the absorber plates contained a Tungsten based alloy. In both cases the tail catcher featured steel absorber plates.
In the various test beam campaigns combined, spanning the years 2010 -- 2012, the DHCAL recorded of the order of 14 Million muon events and 36 Million secondary beam events, where the latter contained a mixture of electrons, muons, pions, and protons. 
\subsection{Current R\&D activities}
The analysis and publication of the test beam results are currently the highest priority of the DHCAL group. Major challenges, such as the calibration (or equalization) of the response of the RPCs and the detailed simulation of the response of RPCs, are very close to having been overcome [7-11]. 
In parallel, to the analysis of test beam data, the group is pursuing the following R\&D activities:
\subsubsection{Development of 1--glass RPCs}
The DHCAL prototype featured a standard chamber design based on RPCs with two resistive plates. The Argonne group, however, proposes to eliminate one of the glass plates in future applications. The advantages are: close to unit pad multiplicity with significant simplification of the calibration and monitoring procedure, reduced thickness of the active element, higher rate capability, and insensitivity of the response to the surface resistivity of the resistive layer (used to apply the High Voltage). To date several 1-glass RPCs have been assembled. The chambers tested very well with cosmic rays. Tests in particle beams are planned for future test beam campaigns.  
\subsubsection{Development of high-rate RPCs}
Due to the high bulk resistivity of glass (and Bakelite), RPCs are notoriously rate limited [12]. The DHCAL group is addressing this shortcoming with the development of semi-conductive glass (in cooperation with COE college) and of low-resistivity Bakelite (in co-operation with USTC). First chambers with samples of low-resistivity glass plates have been assembled and have tested in the Fermilab test beam.
\subsubsection{Development of a High-Voltage distribution system}
With up to 50 layers in a single calorimeter module, a cost-effective way to distribute the High-voltage to individual layers is required. A system capable to regulate the voltage within a few 100 V, to monitor both the current and the voltage, and to switch off individual channels, is being developed. A first prototype controlling a single channel has been assembled and tested successfully with an RPC. The development is currently on hold due to lack of funding.
\subsubsection{Development of a gas recycling system}
The operation of RPCs requires a gas mixture, which is both costly and environmentally harmful. To limit the effect of releasing gas into the environment, the DHCAL group is developing a gas recycling system. The system is based on a new approach, appropriately labeled ``Zero Pressure Containment''. A prototype of the gas collection subsystem is currently being assembled; however, progress is again slow due to lack of funding. 
\subsubsection{Development of the next generation front-end readout system}
The next generation front-end readout system will contain several upgrades compared to the current system: higher channel count, token ring passing, low power operation, power pulsing, and improved internal charge injection systems. To proceed, the project is awaiting funding from both US and Chinese agencies.

\subsection{Engineering challenges}
Several engineering challenges remain to be addressed before an RPC-based DHCAL can be proposed as an option for a colliding beam detector. Following is an (incomplete) list of the major issues:
-   Industrialization of the construction of RPCs. 
-   Design of the readout boards, covering the entire area of the layer (with varying width). The design is expected to feature only a minimum number of different boards. 
-   Design of the gas distribution system, which ensures equal pressure in all layers of a given module, independent of its orientation.
-   Development of a cooling strategy for the front-end boards, which will include power pulsing, as well as active cooling.
-   Development of a module assembly procedure.

\subsection{Plans for the coming years}
The activities of the coming years depend strongly on the progress with the Japanese intentions to host the ILC. Assuming the ILC project goes ahead, the DHCAL group will
a)  Complete the analysis and publication of the test beam data,
b)  Complete the R\&D projects listed above, and
c)  Start the development of the design of calorimeter modules.
In case, the ILC is not going forward, the group plans on completing the data analysis and to continue the tests of high-rate RPCs. Other R\&D projects, such as the development of distribution systems, will be put on hold.
 
\subsection{Applications beyond the ILC}
The DHCAL technology was specifically developed for the hadron calorimeter of the ILC, with its low particle rate and radiation dose. To export the technology to other environments, the rate capability of the chambers and the radiation hardness of the readout need to be improved. The former is being addressed with low-resistivity plates (glass and Bakelite), while the latter will require a new front-end readout system based on an ASIC using a smaller feature size. Possible applications are the tail catcher of the forward calorimeters of CMS and the outer wheels of the ATLAS muon system. Both options are being pursued actively.

References
\begin{enumerate}
\item \fullcite{Drake200788}
\item \fullcite{1748-0221-3-05-P05001}
\item \fullcite{1748-0221-4-04-P04006}
\item \fullcite{1748-0221-4-06-P06003}
\item \fullcite{1748-0221-4-10-P10008}
\item \fullcite{1748-0221-5-02-P02007}
\item CAN-30 and CAN-30A: Analysis of DHCAL Muon Data
\item CAN-31: DHCAL Noise Analysis
\item CAN-32: DHCAL Noise Analysis
\item CAN-39: Analysis of Tungsten-DHCAL Data from the CERN Test Beam
\item CAN-42: The DHCAL Results from Fermilab Beam Tests: Calibration
\item L.Xia: Development of High-Rate RPCs. Talk given at the XII Workshop on Resistive Plate Chambers RPC2014.
\end{enumerate}
