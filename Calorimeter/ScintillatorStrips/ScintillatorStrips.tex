\subsection{Scintillator Strips}
\subsubsection{Collaborating Institutions}
Nihon Dental University, Shinshu University, Tokyo University (ICEPP), Tsukuba University
\subsubsection{Introduction}
\subsubsection{Recent Milestones}
\begin{itemize}
	\item introducing a new scintillation light readout scheme, with different scintillator strip shape  by having better homogeneity 
	\item photo-sensor of increased number of pixels in 1mmx1mm, this leads larger dynamic range for the calorimeter
	\item more experience on the FE read out board and ASICs
\end{itemize}
They are not published yet, instead some proceedings 

\subsubsection{Engineering Challenges}
\begin{itemize}
	\item wrapping the scintillator strip and align them on the FE read out layer automatically 
	\item mass test facility for the read out layer
\end{itemize}

\subsubsection{Future Plans}
\begin{itemize}
	\item optimizing scintillator layer: shape of scintillator strip, how to read out scintillation light, the location of  photo-sensor, size and shape of photo-sensor and mass production scheme
	\item developing photo-sensor with Hamamatsu photonics company, to have lager dynamic range and mass test scheme
	\item establish a detector fabrication plan
\end{itemize}

\subsubsection{Applications Outside of Linear Colliders}
\begin{itemize}
	\item photo-sensor named MPPC from Hamamatsu photonics INC is employed for the T2K experiment, CMS upgrade (HC-CAL), BELLII detector (endocarp muon)
	\item PET and SPECT development 
\end{itemize}
