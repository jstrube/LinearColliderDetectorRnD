\thispagestyle{empty}
\newgeometry{margin=1.5cm} % modify this if you need even more space
\begin{landscape}
    \centering
    \begin{adjustbox}{max width=1\textheight,totalheight=1\textwidth}
\begin{tabularx}{2\textheight}{lXXXX}
    \toprule
    R\&D Technology & Participating Institutes & Description / Concept & Milestones & Future Activities \\
    \midrule
    Scintillator ECAL                                                                                             &
    Nihon Dental University\newline Shinshu University\newline Tokyo University, ICEPP\newline Tsukuba University &                                                                                                                                                                                                                                                                                                                                                                                      &                                                                                                                                                                                                                                                                 &                                                                                                                                                                                                                                     \\
    \midrule
    SiliconECAL ILD &
    LLR / Palaiseau\newline
    LAL / Orsay\newline
    LPNHE / Paris\newline
    University of Tokyo\newline
    Kyushu University\newline
    SKKU / Suwon, Korea\newline
    LPSC / Grenoble\newline
    OMEGA / Palaiseau &                                                                                                                                                                                                                                                                                                                                                                                      High granularity ECAL ($\approx\unit[4000]{channels/dm^3}$). Active sensor: square matrix of about $\unit5\times5]{mm^2}$ PIN diode pixels produced from one high resistivity Si wafer. 4 sensors are glued to PCB holding fully integrated readout electronics and passive cooling. Absorber: self-supporting modular tungsten in carbon-fiber structure. &
    2013--: tests of several layers of technological prototype with one sensor per PCB.\newline
    2014--2015: Design, production and first tests of prototypes with 4 sensor PCBs. Sensors are glued to PCB by a robot. Design of a distributed, quality controlled assembly chain of detector elements. &
    2015-- SPS beam tests of a new several layer prototype. Each layer has one PCB with 4 sensors (1024 pixels of $\unit[5.5\times 5.5]{mm^2}$).\newline
    Documentation of prototype production steps for future industrial mass production.\newline
    Tests of sensors of various designs / manufacturers.\newline
    Optimization of DAQ electronics.\newline
    Design, production and tests of ILD-like detector element with several PCBs connected consecutively and readout from one end. \\
    \midrule
    SiliconECAL SiD &
    &                                                                                                                                                                                                                                                                                                                                                                                      &                                                                                                                                                                                                                                                                 &                                                                                                                                                                                                                                     \\
    \midrule
    AHCAL &
    DESY\newline
    Hamburg\newline
    Heidelberg\newline
    MPI Munich \newline
    Wuppertal\newline
    Mainz\newline
    Omega\newline
    CERN\newline
    ITEP\newline
    MEPHI\newline
    Dubna\newline
    Prague\newline
    NIU\newline
    Tokyo University, ICEPP\newline
    Bergen\newline
    Shinshu &
     The analog hadron calorimeter is based on small plastic scintillator tiles read out with SiPM. It uses fully integrated electronics with power pulsing, auto-trigger and time-stamping capability.                                                                                                                                                                                   &
     2014 - multi-layer test beam campaign at CERN with technical prototype electronics, including large-size layers (4 HBUs)\newline
     2015 - First beam tests of full HBU with SMD SiPMs fabricated with automated assembly procedure                                         &
     2015 Test beams at DESY and SPS with \textgreater 15 HBUs\newline
     2016/17 Test beam at SLAC with 15 layer EM stack, powerpulsing \& ILC time structure, tests in magnetic fieldFurther develop SMD SiPM HBUs, explore ``mega-tile'' options \newline
    Hadronic beam tests with a prototype with ~ $\unit[1]{m^3}$ fully instrumented volume (depends on pending funding request) \\
    \midrule
    DECAL &
    University of Birmingham (inactive) \newline
    University of Bristol (inactive) \newline
    Imperial College London (inactive) \newline
    Queen Mary, University of London (inactive) \newline
    Rutherford Appleton Laboratory (inactive) &
    The digital electronic calorimeter (DECAL) proposes to use monolithic active pixel sensors (MAPS) for the readout of the silicon-tungsten ECAL. The pixels are small enough to count the number of secondary particles of the particle shower, hence the digital calorimeter. &
    Four TPAC 1.2 sensors were tested at CERN (2009) and DESY (2010). The tests validated the INMAPS process and demonstrated that sensors with a high-resistivity epitaxial layer can meet the required MIP efficiency. &
    DECAL efforts are currently dormant. The Arachnid collaboration continues some of the work on MAPS chips. \\
    \midrule
    DHCAL (RPC)	&
    Argonne National Laboratory\newline
    Boston University                  \newline
    COE College (Iowa)                         \newline
    University of Iowa                                 \newline
    Shanghai Jiao Tong University -- SJTU (in discussion)      \newline
    University of Science and Technology of China -- USTC (in discussion) &
     &
     &                                                                                                                                                                                                                                   \\
     \midrule
    SDHCAL (RPC)                                                                                                   &                                                                                                                                         &                                                                                                                                                                                                                                                                                                                                                                                      &                                                                                                                                                                                                                                                                 &                                                                                                                                                                                                                                     \\
    \midrule
    SDHCAL(micromegas) &
     CALICE (LAPP) \newline CEA Saclay\newline Institute of Nuclear and Particle Physics, Demokritos                                                            &
      Micromegas is a thin steel micromesh that separates the gas volume in a region of charge conversion and a region of charge multiplication. It is interesting for EM and H calorimetry because its signal is proportional to the the energy deposit in the gas. To avoid discharge upon very large energy deposits, it now incorporates resistive elements on the readout electrodes. &
       2012: construction and successful test of 1x1 m2 realistic prototypes for the CALICE SDHCAL.\newline
       2014: demonstration of discharge suppression with small resistive prototypes.\newline
       2015: optimisation of the resistive prototypes for best linearity and rate capability. &
       Construction of a Micromegas calorimeter prototype for measuring performance to electrons and later hadrons.                                                                                                                        \\
       \midrule
    GEM DHCAL &
     University of Texas, Arlington &                                                                                                                                                                                                                                                                                                                                                                                      &                                                                                                                                                                                                                                                                 &                                                                                                                                                                                                                                     \\
     \midrule
     Thick Gems &
     Weizmann Institute, Rehovot\newline
     Coimbra University \newline
     Aveiro University &
     Cost-effective sampling element based on the a novel concept derived from the Thick gaseous electron multiplier &
     2014: $\unit[10\times 10]{cm^2}$ discharge free (Ne/\ce{CH4}) single stage RPWELL \newline
     2015: $\unit[30\times 30]{cm^2}$ discharge free (Ar-based gas mixture) &
     Early 2016: new design of $\unit[30\times 30]{cm^2}$ detector\newline
     2016: $\unit[1\times 1]{m^2}$ prototype\newline
     2017- testing RPWELL layer in a fully equipped (S)DHCAL \\
     \midrule
    Dual Readout \newline RD52                                                                                               &
     Texas Tech University\newline Iowa State University\newline INFN (Pavia, Pisa, Cagliari, Rome, Cosenza, Lecce)\newline LIP Lisbon\newline CERN\newline Tufts University &
     Measure scintillation and Cerenkov light independently in optical fibers and measure neutron content event-by-event. Current small modules are dominated by lateral leakage. &
     Twenty-nine papers published in NIM on all aspects of dual readout calorimetry, including crystal dual readout. GEANT (FTFP HP) simulations of a large copper module yield an energy resolution approximately represented by $\sigma/E \approx 30\%/\sqrt{E}$ for pion-induced showers. &
     Measure the difference between pion-induced and proton-induced hadronic showers; measure the time history of light at \unit[5]{GHz}. Build a large module \unit[4]{ton} for final test of hadronic performance. \\
    \bottomrule
\end{tabularx}
\end{adjustbox}
\end{landscape}
\restoregeometry
