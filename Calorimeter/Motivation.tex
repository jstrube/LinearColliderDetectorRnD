\section{Motivation and Constraints for Calorimetry at Linear Colliders}

The design of Linear Collider experiments is fundamentally influenced by the requirement for excellent jet energy resolution. The goal of being able to reconstruct hadronic \PZ events with a resolution comparable to the intrinsic width translates to $\approx 3.5\%-5\%$ jet energy resolution.

Linear Collider Detector concepts are designed for particle flow, which exploits the fact that approximately 60\% of the energy in a typical jet is in the form of charged tracks, 30\% is in form of photons, and 10\% occurs as neutral hadrons. With the tracking detectors having the best energy resolution, and hadronic calorimeters having the lowest, the reconstruction performance on jets depends crucially on being able to separate the different components of a jet. To avoid energy loss in insensitive material, detector concepts at Linear Colliders are designed with calorimeters inside the solenoid. This requires the absorber material to have a very short interaction length to contain the shower.

Highly granular calorimeter concepts consist of segmented absorber stacks interspersed with measurement layers, with $\text{mm}^2$ segmentation in the electromagnetic calorimeter and $\text{cm}^2$ in the hadronic calorimeter. Digital readout requires an order of magnitude smaller segmentation and is also being studied. These calorimeters aim at separating the showers of individual particles in situ, and at matching the detector signatures of particles across the main tracker, the electromagnetic calorimeter, and the hadronic calorimeter. Their performance relies in principle on the ability to subtract the contributions of charged hadrons from the calorimeters.

Dual readout calorimeters aim at containing particles in large crystals, and distinguishing electromagnetic and hadronic contributions by measuring scintillation and Cherenkov radiation separately.
