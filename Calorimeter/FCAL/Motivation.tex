\section{Motivation and Constraints for Forward Calorimetry at Linear Colliders}

While radiation levels in a linear collider detector are generally small compared to the conditions at hadron colliders. However, to increase angular coverage, concepts are instrumented with detectors close to the beam line. In addition to improving missing energy measurements by extending the angular coverage, a whole class of SUSY models predicts new physics that would occur predominantly at very low angles.
 Detectors in this region require specialized designs to cope with the radiation levels while targeting high granularity and the extraordinary physics performance that is required to separate new physics signals from the large number of electromagnetic processes close to the beam.
 Furthermore, these forward calorimeters can be used as beam feedback systems, that augment the accelerator beam instrumentation.
