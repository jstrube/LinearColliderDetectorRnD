\section{LCSim}
\subsection{Collaborating Institutions}

The core software has been developed at SLAC. A number of packages were
contributed by university and other national lab groups when such efforts were
supported by DOE.

\subsection{Introduction}
\subsection{Recent Milestones}

Simulation of physics processes and detector response is crucial to the design
of new HEP experiments such as those proposed for the ILC. There are stringent
requirements on the design of tools for detector R&D which differentiate them
from typical experiment-specific simulation and reconstruction code. They must:
\begin{itemize}
\item allow easy reconfiguration to support different detector geometries and technologies,
\item make it easy to develop, implement and compare new reconstruction algorithms,
\item be very easy for users to set up and quickly become productive with,
\item work on a wide variety of operating systems and computing platforms,
\item be easy to develop and support using a fraction of the manpower that would be available to an established experimental collaboration.
\end{itemize}
The lcsim physics and detector response simulation and event reconstruction
toolkit was developed at SLAC to provide a flexible and performant suite of
software programs to allow fast and efficient studies of multiple detector
designs for the ILC. The primary goal of the group has been to develop computing
infrastructure to allow physicists from universities and other labs to quickly
and easily conduct physics analyses and contribute to detector R&D. These tools
include the Geant4-based detector response simulation program (slic), and the
Java-based reconstruction and analysis tools (org.lcsim) [1].

\subsection{Engineering Challenges}

The lcsim software pioneered the use of runtime-defined detector geometries.
Although LCIO [2] provided a common event data model for all ILC groups, the
community was never able to agree upon a common geometry system for both
simulation and reconstruction. DD4hep [3] is an effort within the AIDA Common
Software Tools project to provide such functionality. Although it adopted many
lcsim concepts, the implementation of the software has taken its own course. The
largest challenge to the lcsim effort at the moment (beyond lack of funding) is
to maintain some connection to the geometry definition functionality promised by
DD4hep.

\subsection{Future Plans}

The core functionality is being kept current by upgrading to the latest versions
of Geant4, etc. Due to lack of funding, the project is currently primarily
responding to user requests for additional functionality.

\subsection{Applications Outside of Linear Colliders}

The flexibility and power of this simulation package make it not only useful for
the application domain for which it was developed (viz. HEP collider detector
physics), but also for other physics experiments, and could very easily be
applied to other disciplines, e.g. biomedical or aerospace, to efficiently use
the full power of the Geant4 toolkit to simulate the interaction of particles
with fields and matter. The Heavy Photon Search experiment [4] at Thomas
Jefferson National Laboratory has adopted slic as its detector response
simulation package and the org.lcsim toolkit for its event reconstruction needs.
Physics and detector studies for CLIC [5] and the Muon Collider [6] have also
used both slic and the org.lcsim software. The software could be easily used for
physics and detector studies at detectors at future circular colliders.


[1] http://www.lcsim.org
[2] http://lcio.desy.de/
[3] http://aidasoft.web.cern.ch/DD4hep
[4] https://confluence.slac.stanford.edu/display/hpsg/Heavy+Photon+Search+Experiment [5] http://clicdp.web.cern.ch/
[6] http://map.fnal.gov/
