\section{LCIO}

\subsection{Introduction}

\subsection{Recent Milestones}
The LCIO software toolkit~\cite{lcioWebsite} was developed to provide a common event data model
(EDM) and persistency format for Linear Collider physics and detector
simulations. It was developed as a joint effort between SLAC and DESY and has
been adopted by all of the detector concepts for both ILC and CLIC. Many of
the sub-detector R\&D groups (e.g. CALICE and LCTPC) have also adopted LCIO for
both their simulation needs and for testbeam data. Major R\&D Efforts: The
software toolkit consists of an Application Programming Interface (API) as well
as reference implementations in Java and C++ and a binding to python. This, plus
its deliberately simple design and well-documented EDM, has allowed the LC
community to mix-and-match its software applications. Events simulated in C++
can be reconstructed in Java and analyzed in python, providing enormous
flexibility to the end user, who can concentrate on analyses and not be hampered
by programming language limitations.

\subsection{Engineering Challenges}
The implementation of a complete EDM for all HEP applications is very difficult,
if not impossible. LCIO has succeeded by reducing the problem to its simplest
solution, but providing end users flexibility to adapt to their specific needs.
Custom classes can be implemented using extensions to existing classes or using
generic objects and relations between collections of data. The LCIO development
team has also added functionality as new classes have been requested by users.
Maintaining strict control over the structure of the LCIO EDM and persistency
has ensured that any LCIO file can be opened and interpreted without having
access to the code which created it.

\subsection{Future Plans}
Continued development of LCIO will be driven by user demand and developers' resources.

\subsection{Applications Outside of Linear Colliders}
The Heavy Photon Search experiment at Thomas Jefferson National Laboratory
has adopted LCIO as its event data model and data persistency format. Physics
and detector studies for CLIC and the Muon Collider have also used LCIO.
The authors of the Whizard~\cite{whizardWebsite} event generator have expressed interest in using
LCIO as their binary persistency format for Monte Carlo events. This could lead
to its integration into other experiments. Because of its simple and
well-documented persistency format LCIO is a perfect candidate for HEP data
archiving applications.
