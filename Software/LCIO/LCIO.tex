\section{LCIO}

\subsection{Introduction}

The LCIO software toolkit~\cite{lcioWebsite} provides an event data model
(EDM) and persistency format for physics and detector
simulations. It was developed as a joint effort and has
been adopted by all of the detector concepts for both ILC and CLIC. Many of the sub-detector R\&D groups (e.g. CALICE and LCTPC) have also adopted LCIO for both their simulation needs and for testbeam data. The
software toolkit consists of an Application Programming Interface (API) with implementations in Java and C++ and a binding to python.
\subsection{Recent Milestones}
{\color{red} Please provide}
\subsection{Engineering Challenges}
{\color{red} What are the engineering challenges \emph{that will need to be solved} when the ILC becomes real.}

\subsection{Future Plans}
Continued development of LCIO will be driven by user demand and developers' resources.

\subsection{Applications Outside of Linear Colliders}
The Heavy Photon Search experiment at Thomas Jefferson National Laboratory
has adopted LCIO as its event data model and data persistency format. Physics
and detector studies for CLIC and the Muon Collider have also used LCIO.
The Whizard~\cite{whizardWebsite} event generator uses LCIO as a possible output format for Monte Carlo events. This could lead
to its integration into other experiments. Because of its simple and
well-documented persistency format LCIO is a perfect candidate for HEP data
archiving applications.
