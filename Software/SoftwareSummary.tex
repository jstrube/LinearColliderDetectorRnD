\thispagestyle{empty}
\newgeometry{margin=1.5cm} % modify this if you need even more space
\begin{landscape}
    \centering
    \begin{adjustbox}{max width=1.3\textwidth,totalheight=1.\textheight}
\begin{tabularx}{1.2\textheight}{lXXXX}
    \toprule
    R\&D Technology & Participating Institutes & Description / Concept & Milestones & Future Activities \\
    LCIO & SLAC, DESY & The LCIO software toolkit provides a common event data model (EDM) and persistency format for Linear Collider physics and detector simulations. & & \\
    LCSim & SLAC & The software programs in the lcsim physics and detector response simulation and event reconstruction toolkit allow fast and efficient studies of multiple detector designs for the Linear Collider. & & \\
    DD4HEP & CERN & DD4hep provides a generic, consistent and complete detector description, including geometry, materials, visualization, readout, alignment and calibration. It supports the full experiment life cycle: from detector concept development over detector optimization and construction to the operation phase. & & \\
    Marlin & DESY & Marlin is a C++ application framework for processing LCIO event data files. & & Parallelization and adaptation to multi-core hardware \\
    PandoraPFA & Cambridge & The PandoraPFA software package consists of a robust and efficient C++ software development kit (SDK) and
    libraries of reusable pattern-recognition algorithms that exploit functionality
    provided by the SDK. Algorithms have been developed to provide a particle flow
    reconstruction of events in fine-granularity detectors, such as those proposed
    for use at the Linear Collider. & & \begin{itemize} \item improvement
    of $\pi^0$ reconstruction \item improve the ability to identify and
    separate neutral hadrons from nearby charged hadrons. \end{itemize} \\
    % \midrule
    LCFIPlus &
    Kyushu \newline
    Tokyo \newline
    PNNL \newline &
    LCFIPlus consists of algorithms for vertex reconstruction, jet clustering, and flavor tagging. &
    Development of a vertex-aware jet finder\newline
    Efficient flavor tagging in 6-jet events &
    Improvements to vertex reconstruction \newline
    Addition of particle ID to flavor tagging \newline
    Better integration with geometry description \\
    \bottomrule
\end{tabularx}
\end{adjustbox}
\end{landscape}
\restoregeometry
