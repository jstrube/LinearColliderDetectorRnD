\section{Motviation and Constraints for Vertex Detectors at Linear Colliders}

The reconstruction of displaced decays has been an important part of particle physics programs since the days of bubble chambers and the discovery of ``V'' particles. This is still true in today's high-energy collider experiments. If long-lived particles, such as B or D mesons, or tau leptons, decay to at least one charged track in the detector, they can in principle be resolved from the interactions of the primary collision. Similarly, the possibility to distinguish several primary interaction points significantly improves the reconstruction of events. To this end, modern vertex detectors use silicon sensors with small pixels, assembled in structures with as few radiation lengths as possible.

Highly performing vertex detectors are essential for the succes of the physics program at the ILC. Discovery channels for a large class of new physics models involve third-generation fermions: b quarks that form long-lived mesons, top quarks that decay predominantly to a b quark and a W boson, and long-lived tau leptons. Furthermore, highly performing reconstruction of displaced vertices allows the distinction between the decays of B and D mesons and thus the reconstruction of the decay $\PH\to\PQc\PAQc$, which is not possible at the LHC experiments. The resolution of the perigee, or \emph{impact parameter} of a helical track from the interaction point can be parameterized as
\begin{equation}
	\sigma_\text{ip} = \left(\frac{\alpha}{p^{3/2}\sin\theta}\right)
\end{equation}
The goal for a vertex detector in an ILC experiment is XXX
