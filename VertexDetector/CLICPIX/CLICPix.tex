\section{CLICPix}
% \subsection{Collaborating Institutions}
% \begin{itemize}
%     \item CERN
%     \item Spanish network for Future Linear Colliders
%     \item University of Liverpool
%     \item Institute of Space Science, Bucharest
%     \item University of Bristol
% \end{itemize}
% The University of Glasgow and the University of Oxford also intend to be involved, but they have not yet contributed.

\subsection{Introduction}
To achieve the physics goals of flavour tagging at CLIC, a vertex pixel detector with high spatial precision (\unit[3]{\micron} single-point resolution), \unit[10]{ns} time stamping and ultra- low mass (0.2\% X0 per detection layer) will be required.

\subsection{Recent Milestones}
\begin{itemize}
\item Development of the CLICpix hybrid pixel readout ASIC with \unit[25]{\micron} pitch, analog readout, time stamping, and power-pulsing functionality, implemented in \unit[65]{nm} CMOS technology
\item Development of ultra-thin (\unit[50]{\micron}) planar pixel sensors, as well as active sensors with capacitive coupling
\item Low-mass fine-pitch interconnects between sensor and ASIC
\item Through-silicon via technology for powering, configuration and readout of the ASIC
\item Low-mass powering infrastructure, including power-pulsing with local energy storage
\item Low-mass carbon-fibre supports
\item Detector cooling based on forced air-flow
\item Concepts for mechanical integration and detector assembly
\item Detector layout optimisation studies
\end{itemize}

\subsection{Engineering Challenges}
