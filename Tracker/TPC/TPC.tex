\section{Time Projection Chamber -- Bonn}
\subsection{Introduction}
The University of Bonn is studying the pixelized readout of a TPC for the ILD detector. The readout is based on the Timepix ASIC with a triple GEM or Micromegas based gas amplification.

\subsection{Recent Milestones}
The first studies were based on the triple GEM setup with a single Timepix chip. This readout was mounted in a small test detector in the Bonn laboratory. Here, the working principle was tested with a long drift distance. It could be demonstrated that the transverse spatial resolution of the reconstructed primary electrons was close to the expected diffusion limit of single electrons. The results are summarized in the following publications:
\begin{itemize}
\item \fullcite{6359808}
\item \fullcite{4774978}
\item \fullcite{1748-0221-4-11-P11015}
\item \fullcite{Kaminski:2010zzc}
\item \fullcite{Schade2011128}
\end{itemize}

The new focus are GridPix based detectors, where the gas amplification stage is a Micromegas produced in a postprocessing technique, which guarantees a high quality grid well aligned with the readout pixels. This approach was pioneered by NIKHEF and the University of Bonn has modified the production process together with the Fraunhofer Institut IZM so that a wafer-based production of GridPix detectors is standard by now. The new GridPixes were tested on small prototype detectors and also assembled in an 8 GridPix module for the Large Prototype detector at DESY. A successful test beam campaign was performed last year.
\begin{itemize}
\item \fullcite{1748-0221-9-01-C01033}
\item \fullcite{Koppert2013245}
\end{itemize}
The current work is focused on a new LP module with about 100 GridPixes. This module is a demonstrator that larger areas (~\unit[400]{$\mathrm{cm}^2$}) can be produced and operated. It shall be tested in the LP at the beginning of next year. For this a number of challenges have to be coped with. In particular commercial readout systems are not easily scalable. This is why Bonn has developed a cheap and easily expandable system based on the Scalable Readout System (SRS) of the RD51 collaboration.
In addition Bonn is developing the software for reconstructing and analyzing the test beam and simulation data. For this the LCTPC software framework of MarlinTPC is used.
\begin{itemize}
\item \fullcite{4774731}
\end{itemize}
Finally, Bonn also takes part in designing new pixel chips. To test the new digitization and readout techniques two test chips were designed in collaboration with N'IKHEF. Then Bonn also contributed to the design of the Timepix successor chip, Timepix3, which is being tested now:
\begin{itemize}
\item \fullcite{1748-0221-5-12-C12005}
\item \fullcite{6748097}
\item \fullcite{1748-0221-9-01-C01052}
\end{itemize}

\subsection{Engineering Challenges}
The production of a module with 100 GridPixes requires 4 main components: The production of a
large number of GridPixes with sufficiently good quality. This has been addressed by the new production method and a large batch is being produced. The challenge of the readout is being addressed by the new readout system. Finally the distribution of the LV power to all ASICs and the cooling of the ASICs still are unclear, but since both challenges are similar for most readout electronics, standard solutions are expected to be adequate.

\subsection{Future Plans}
On a short term the production of the 100 ASIC module is the main goal at Bonn. If this module has been successfully operated, we are interested in replacing the Timepix ASIC by the Timepix3 ASIC and produce GridPix detectors with this improved chip. There are also some ideas of how to improve the grid structure and make it more reliable. Finally, the reconstruction and analysis software needs further improvement and has to be extended, so that simulated data for the final TPC (i.e. 10,000 hits per track) can be studied.

\subsection{Applications Outside of Linear Colliders}
A single InGrid detector will be installed this year in the CAST experiment for axion search. For a CLIC--TPC a highly granular (i.e. pixelized) readout structure is mandatory to lower the occupancy.
