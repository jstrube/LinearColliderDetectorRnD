\thispagestyle{empty}
\newgeometry{margin=1.5cm} % modify this if you need even more space
\begin{landscape}
    \centering
    \begin{adjustbox}{max width=1.1\textwidth,totalheight=1\textheight}
\begin{tabularx}{1.6\textheight}{llXXX}
    \toprule
    R\&D Technology & Participating Institutes & Description / Concept & Milestones & Future Activities \\
    \midrule
        SiPM &
        DESY &
        The main option for the sensitive layers will use extruded plastic scintillation strips, composed of polystyrene doped with 1.0\% PPO and 0.03\% POPOP.
        The signals will be readout from both sides of the strips by Silicon photomultipliers, coupled to the wave length shifting (WLS) fibres.
        Reading out both sides of a strip offers the possibility to define the position of the hits along the strip, which will help in reducing the fake rate in the muon system. &
\begin{enumerate}
\item Detailed Full Monte Carlo Simulation of the Muon System/Tail Catcher, it is concern to choose the geometry of detection plane, geometry of the stereo layers, number of the layers of the detection system and their position, especially concern to the energy leakage,
\item Study of the performance of the Muon System/Tail Catcher with framework of PFA,
\item Optimization of the detection elements of the Muon System/Tail Catcher. The first priority is the design of the main detection elements a scintillation strips with the SiPM readout,
\item Development of the Digital options of the SiPM for the Muon /Tail Catcher System, which will dramatically simplify the readout electronics and data processing.
\end{enumerate} &
\begin{itemize}
\item The performance of the muon system has been evaluated by simulating events in the ILD concept. To determine the optimal layout,
a geometry was created in MOKKA [3] with 19 layers in the barrel and 18 layers in the endcap, at equal distances of \unit[140]{mm}. After simulating the detector response
in  MOKKA once, different layouts could be studied by including or excluding layers in the reconstruction phase. For this study the muon layers have been segmented in pads of $30\times\unit[30]{mm^2}$.
\item Study of the Integration of the Muon System/Tail Catcher in to Solenoid Yoke,
\item Build of the prototypes of the Muon System/Tail Catcher detection elements on base of the Scintillator Strip/Wavelength Shifter and Analog SiPMs for the study of the main elements as thickness, length, reflection coating, wavelength shifters light yield and other. For this purposes will be develop the test setup for the cosmic muons detection,
\item Design and Technology development of the Analog SiPMs on base CMOS technology as preliminary options for the Muon System/Tail Catcher optimization,
\item Technology Development of the Digital option of the SiPMs on base innovative 3D interconnection (3D-IC) technology with fully digital readout and processing electronics.
\end{itemize}
 \\
    \midrule
        RPC &
        DESY, Hamburg &
        Resistive plate chambers (RPC) are considered as an alternative sensitive elements for the Muon/Tail Catcher Detector.&
        &
        \begin{enumerate}
        \item Measure and predict the aging characteristics of the Bakelite RPCs to establish them as alternatives to standard glass or Bakelite RPCs,
        \item Gain experience with glass RPCs while studying possible aging mechanisms. Study of alternative RPC gases could identify gas mixes that would minimize the production of harmful contaminants which lead to premature chamber aging,
        \item Validate the use of the KPIX front-end chip for RPC readout.
        \end{enumerate}
        \\
    \midrule
        Scintillator strips &
        FNAL \newline
        Vinca &
        &
        \\
    \bottomrule
\end{tabularx}
\end{adjustbox}
\end{landscape}
\restoregeometry
