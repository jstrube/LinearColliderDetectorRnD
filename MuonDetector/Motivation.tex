\section{Motivation and Constraints for Muon Detectors at Linear Colliders}

The goal of the muon detectors in the Linear Collider Detector concepts is the identification of minimally ionizing particles. A large fraction of muons (and some pions) do not get stopped in the calorimeters and traverse the solenoid. To measure such charged particles the solenoid field return yoke is instrumented with detectors that measure ionizing particles. The excellent track momentum resolution of the inner systems of Linear Collider Detectors, i.e.\ the trackers and calorimeters means that the main requirement of the muon systems is to be able to match their track segments with those of the inner detectors to be able to tag a track as a muon. In terms of design, this requirement, combined with the low number of muons per event, translates to high efficiency and moderate segmentation.
