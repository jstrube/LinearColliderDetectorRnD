\subsection{Silicon Pads}
\subsubsection{Collaborating Institutions}
If I put a cut on group with ≤1 FTE on 
It remains only LPNHE-Paris, LAL, Univ. of Tokyo and Kyushu University
\subsubsection{Introduction}
\subsubsection{Recent Milestones}
ELECTROMAGNETIC Calorimeter
\begin{itemize}
	\item First development of PFA for dedicated detector (TESLA Report)
	\item First prototype of High granularity electromagnetic calorimeter (“physics prototype”, see publications in the CS report).
	\item First design of ECAL silicon – tungsten for a full scale detector (From TESLA report to DBD 2013)
	\item R\&D on scalable technology for all the involved large detector aspects (integration of embedded readoud chips, on thin supporting electronics boards, in self-supporting tungsten-Carbon mechanical elements ensuring the cooling and protection; all made of exchangeable elements with a quality control procedure; the associated DAQ).
	\item Realisation of a large self-supporting W-Carbon fiber structure with integrated stress monitoring (using Fiber Bragg Gratting) 
	\item Recently: tests of 1st base sensor units of the technological prototype in beam
\end{itemize}
PFA:
\begin{itemize}
	\item Development of Mokka an overlayer of the GEANT4 used for ILD, CLIC detectors, CALICE TB, …
	\item Reconstruction tools adapted to the high granularity calorimeters (photon reconstruction [GARLIC], Advanced clustering [ARBOR], event displays [DRUID])
\end{itemize}

ILD integration \& optimisation
\begin{itemize}
	\item for the DBD: integration of all the ILD elements, placement of services, thorough estimation of total cost of the detector
	\item since DBD: re-optimisation of the ILD dimensionning, esp. for the Si-W ECAL using full PFA reconstruction.
\end{itemize}

\subsubsection{Engineering Challenges}
\begin{itemize}
	\item Silicon wafer cost reduction when used for calorimetry; direct contact with producers established (Hamamatsu, On-Semi, …).
	\item A chip with the good dynamic, noise, power dissipation (using power pulsing), etc;..
	\item Integration in a compact device, ensuring all the requests (precision: electronic and mechanic, heat production, reliability)
	\item Industrialisability of solutions; scalability of tests for a 100M channel detector.
\end{itemize}
\subsubsection{Future Plans}
Impossible. No way to see beyond next year (see IN2P3 recommendation)
To recall, all the R\&D will stop at the end of 2016, if there is no decision in Japan
\subsubsection{Applications Outside of Linear Colliders}
\begin{itemize}
	\item CEPC, TLEP and directly today on CMS upgrade
	\item The compact Silicon-W design has been used in the PAMELA satellite (very similar to physics prototype)
\end{itemize}
