\subsection{Silicon Pads}
\subsubsection{Collaborating Institutions}
If I put a cut on group with ≤1 FTE on 
It remains only LPNHE-Paris, LAL, Univ. of Tokyo and Kyushu University
\subsubsection{Introduction}
\subsubsection{Recent Milestones}
ELECTROMAGNETIC Calorimeter
> First development of PFA for dedicated detector (TESLA Report)
> First prototype of High granularity electromagnetic calorimeter (“physics prototype”, see publications in the CS report).
> First design of ECAL silicon – tungsten for a full scale detector (From TESLA report to DBD 2013)
> R\&D on scalable technology for all the involved large detector aspects (integration of embedded readoud chips, on thin supporting electronics boards, in self-supporting tungsten-Carbon mechanical elements ensuring the cooling and protection; all made of exchangeable elements with a quality control procedure; the associated DAQ).
> Realisation of a large self-supporting W-Carbon fiber structure with integrated stress monitoring (using Fiber Bragg Gratting) 
> Recently: tests of 1st base sensor units of the technological prototype in beam
PFA:
> Development of Mokka an overlayer of the GEANT4 used for ILD, CLIC detectors, CALICE TB, …
> Reconstruction tools adapted to the high granularity calorimeters (photon reconstruction [GARLIC], Advanced clustering [ARBOR], event displays [DRUID])
ILD integration \& optimisation
> for the DBD: integration of all the ILD elements, placement of services, thorough estimation of total cost of the detector
> since DBD: re-optimisation of the ILD dimensionning, esp. for the Si-W ECAL using full PFA reconstruction.

\subsubsection{Engineering Challenges}
>  Silicon wafer cost reduction when used for calorimetry; direct contact with producers established (Hamamatsu, On-Semi, …).
>  A chip with the good dynamic, noise, power dissipation (using power pulsing), etc;..
> Integration in a compact device, ensuring all the requests (precision: electronic and mechanic, heat production, reliability)
> Industrialisability of solutions; scalability of tests for a 100M channel detector.
\subsubsection{Future Plans}
> Impossible. No way to see beyond next year (see IN2P3 recommendation)
To recall, all the R\&D will stop at the end of 2016, if there is no decision in Japan
\subsubsection{Applications Outside of Linear Colliders}
> CEPC, TLEP and directly today on CMS upgrade
> The compact Silicon-W design has been used in the PAMELA satellite 
(very similar to physics prototype)
